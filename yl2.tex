\section{Ülesanne 2.}

\subsection*{}

Tõestame teoreemi kahe teguri korrutise kohta, millest järeldub teoreemi kehtivus.
Olgu $\log_a b_1 = x_1$ ja $\log_a b_2 = x_2$, siis $b_1 = a^{x_1}$ ja $b_2 = a^{x_2}$. Leiame arvude $b_1$ ja $b_2$ korrutise: $b_1 b_2 = a^{x_1} a^{x_2}$ ehk $b_1 b_2 = a^{x_1 + x_2}$. Logaritmi definitsiooni järgi saame viimasest võrdusest, et $\log_a(b_1 b_2) = x_1 + x_2$. Asendades $x_1$ ja $x_2$ vastavate logaritmidega, saame:

\begin{equation}
	\log_a(b_1 b_2) = \log_a b_1 + \log_a b_2
\end{equation}

\subsection*{}

Kahe nurga vahe ja summa tangensi valemite tuletamiseks kasutame ühe ja sama nurga trigonomeetriliste funktsioonide vahelisi põhiseoseid ja eespool saadud valemeid:

\begin{equation}
	\tan(\alpha	- \beta)
		= \frac{\sin(\alpha	- \beta)}{\cos(\alpha	- \beta)}
		= \frac{\sin\alpha\cos\beta - \cos\beta\sin\alpha}
					 {\cos\alpha\cos\beta + \sin\alpha\sin\beta}
\end{equation}

\subsection*{}

Lahenda võrrandisüsteem
\begin{equation}
  \begin{array}{*{7}{r}}
    x  & + & 2y & + &  z = &  5, \\
    4x & - &  y & + & 2z = & -3, \\
    2x & + & 3y & + & 4z = &  3. \\
  \end{array}
\end{equation}
Lahendus.
\begin{equation}
	D = 
	\begin{array}{|*{3}{r}|}
	1 & 2 & 1 \\
	4 & -1 & 2 \\
	2 & 3 & 4 \\
	\end{array}
	= -4 +12 +8 +2 -32 -6 = -20.
\end{equation}
