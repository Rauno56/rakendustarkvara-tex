\newcommand{\cOut}[2]{
\stackrel{\textstyle #2}{\cancel{#1}}
}

\newcommand{\cOutB}[2]{
\underset{\textstyle #2}{\cancel{#1}}
}

\section{Ülesanne 4.}

\subsection*{}

\begin{displaymath}
  A = \lim_{\Delta x \to 0} \sum_{x=0}^{h} 9810 \pi r^2 x \: \Delta x \: \mathrm{J}.
\end{displaymath}

\subsection*{}

\begin{displaymath}
  a^m : a^n =
  \frac{a^m}{a^n} =
  \frac{\overbrace{aa \dots a}^\text{$n$ tegurit} \cdot \overbrace{aa \dots a}^\text{$m-n$ tegurit}}
  {\underbrace{aa \dots a}_\text{$n$ tegurit}} =
  a^{m-n}.
\end{displaymath}

\subsection*{}

\begin{displaymath}
  \begin{split}
    & S = \int\limits_0^9 3\sqrt{x} \mathrm{d}x - \int\limits_0^9 x \mathrm{d}x =
    3 \int\limits_0^9 x^{\frac{1}{2}} \mathrm{d}x - \int\limits_0^9 x \mathrm{d}x =
    3\left[\frac{x^{\frac{3}{2}}}{\frac{3}{2}}\right]_0^9 - \left[\frac{x^2}{2}\right]_0^9 = \\
    & 2[x \sqrt{x}]_0^9 - \frac{1}{2} [x^2]_0^9 =
    2 \cdot 27 - \frac{1}{2} \cdot 81 = 54 - 40,5 = 13,5 \: \text{ruutühikut}.
  \end{split}
\end{displaymath}

\subsection*{}
  Näiteks $15^4 : 5^4 = (15 : 5)^4 = 3^4 = 81$.
\par
Viimase tehte õigsus on eriti hästi näha, kui kirjutada jagamine murruna ja siis murdu taandada.
\begin{displaymath}
  \frac{15^4}{5^4} = \frac{\cOut{15}{3} \cdot \cOut{15}{3} \cdot \cOut{15}{3} \cdot \cOut{15}{3}}{\cOutB{5}{1} \cdot \cOutB{5}{1} \cdot \cOutB{5}{1} \cdot \cOutB{5}{1}} = 3^4 = 81.
\end{displaymath}
