\section{Ülesanne 8.}

\subsection{Teoreem}

\begin{definition}[Varjatud Markovi ahel]%~\citep{lember10}
	Protsessi $X = \{Y_t\}_{t\geq1}$ nimetatakse varjatud Markovi ahelaks kui kehtib:

	\begin{enumerate}
		\item $\{Y_t\}_{t\geq1}$ korral, juhuslikud suurused $\{X_t\}_{t\geq1}$ on omavahel sõltumatud;
		\item iga $t = 1,2,\dots$, korral on $X_t$ sõltuv juhuslikust protsessist $\{Y_t\}_{t\geq1}$ (ja ajast $t$) ainult läbi $Y_t$.
	\end{enumerate}

	Juhuslike protsesside paarile $(X, Y)$ viidatakse ka kui \textit{varjatud Markovi mudelile}.
\end{definition}

\subsection{Tabel}

\begin{table}[h]
	\centering
	\begin{tabular}{ r | *{4}{c} }	
						& SV 	& sport & sots. & nutt\\
		\hline
			 kurb & 0,3 & 0,15	 & 0,05 & 0,5 \\
		õnnelik & 0,2	& 0,2   & 0,5	& 0,1 \\
	\end{tabular}

	\caption{Emissioonitõenäosused.}
	\label{fig:HMM.ex.emission}
\end{table}

\subsection{Joonis}

Varjatud Markovi ahela mõistet võib kujutada ka järgneva skeemiga:

\begin{figure}[h]
	\centering
	\input{markovidef.fig}
	\caption{Varajtud Markovi ahela kuju.}
	\label{fig:HMM}
\end{figure}

Joonisel~\ref{fig:HMM} ristküliku sees olev osa on meile üldjuhul vaadeldamatu ja sealt tuleb varjatud Markovi ahelatele ka nimi.
