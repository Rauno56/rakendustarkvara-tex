\section{Ülesanne 8.}

\subsection{Teoreem}

\begin{definition}[Varjatud Markovi ahel]%~\citep{lember10}
	Protsessi $X = \{Y_t\}_{t\geqslant1}$ nimetatakse varjatud Markovi ahelaks kui kehtib:

	\begin{enumerate}
		\item $\{Y_t\}_{t\geqslant1}$ korral, juhuslikud suurused $\{X_t\}_{t\geqslant1}$ on omavahel sõltumatud;
		\item iga $t = 1,2,\dots$, korral on $X_t$ sõltuv juhuslikust protsessist $\{Y_t\}_{t\geqslant1}$ (ja ajast $t$) ainult läbi $Y_t$.
	\end{enumerate}

	Juhuslike protsesside paarile $(X, Y)$ viidatakse ka kui \textit{varjatud Markovi mudelile}.
\end{definition}

\subsection{Tabel}

\begin{table}[h]
	\centering
	\begin{tabular}{ r | *{4}{c} }	
						& SV 	& sport & sots. & nutt\\
		\hline
			 kurb & 0,3 & 0,15	 & 0,05 & 0,5 \\
		õnnelik & 0,2	& 0,2   & 0,5	& 0,1 \\
	\end{tabular}

	\caption{Emissioonitõenäosused.}
	\label{fig:HMM.ex.emission}
\end{table}

\subsection{Joonis}

Varjatud Markovi ahela mõistet võib kujutada ka järgneva skeemiga:

\begin{figure}[h]
	\centering
	\begin{tikzpicture}[
		->,
		>=stealth',
		shorten >=1pt,
		auto,
		node distance=2cm,
		main node/.style={inner sep=0cm,circle,minimum width=1cm,fill=white,draw,font=\small}]

	\node[main node] (y1) {\dots};
	
	\node[main node] (y2) [right of=y1] {$Y_{k-1}$};
	\node[main node] (x2) [below of=y2] {$X_{k-1}$};
	
	\node[main node] (y3) [right of=y2] {$Y_{k}$};
	\node[main node] (x3) [below of=y3] {$X_{k}$};
	
	\node[main node] (y4) [right of=y3] {$Y_{k+1}$};
	\node[main node] (x4) [below of=y4] {$X_{k+1}$};
	
	\node[main node] (y5) [right of=y4] {\dots};
	
	\node[rectangle, inner sep=2mm,draw=black!100, fit=(y1) (y2) (y3) (y4) (y5)] {};

	\path
		(y1) edge node {} (y2)
		(y2) edge node {} (y3)
		(y3) edge node {} (y4)
		(y4) edge node {} (y5)
		
		(y2) edge node {} (x2)
		(y3) edge node {} (x3)
		(y4) edge node {} (x4)
	;
\end{tikzpicture}

	\caption{Varajtud Markovi ahela kuju.}
	\label{fig:HMM}
\end{figure}

Joonisel~\ref{fig:HMM} ristküliku sees olev osa on meile üldjuhul vaadeldamatu ja sealt tuleb varjatud Markovi ahelatele ka nimi.

\subsection{Kirjanduse loetelu}

\begin{thebibliography}{99}

\bibitem{Knuth84} D. E. Knuth. The {\TeX}book.
Addison-Wesley, 1984.

\bibitem{lamport94} Leslie Lamport,
	\emph{\LaTeX: A Document Preparation System}.
	Addison Wesley, Massachusetts,
	2nd Edition,
	1994.

\bibitem{thebestbookintheworld} Eno Raud,
  Sipsik.
  Eesti raamat,
  1962.
\end{thebibliography}
