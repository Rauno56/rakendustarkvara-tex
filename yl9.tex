
\section{Ülesanne 9.}

Idee mida \TeX{} kannab on hea. Mulle meeldib, et saan tavalises tekstivormis oma soovi kirjeldada ning oman seega paremat kontolli tulemuse üle ning olen sarnaseid vahendeid kasutanud ka varajasemalt nii palju kui võimalik(Mark\-downi formaadis).

Samuti kiidan filosoofiat vormistuse korra ja reeglite aspektist, mida \TeX{} justkui "peale surub". Kui tõstame esile sisu ning laseme vormil olla täiesti eraldatud võidavad kõik - nii loov kui tarbiv pool. Sinna suunas liigub ka kogu ülejäänud loov kammuun. Veebiarendus on hea näide.

Teisalt aga leian, et see tööriist on vanamoodne ning ehk aegunudki, kuigi hea alternatiiv puudub. Makrod on kasutamatud ning loetamatu süntaksiga, paketimajanduse haldamine ja konfigureerimine kaootiline, tihti viletsa dokumentatsiooniga ning "automaagilisel" viisil töötav. Süsteemi ülesehitus on kohmakas ja platvorm ise suur ning takistab suuresti normaalset arengut (nagu ma aru saan pakendatakse kõik lisapaketid/moodulid, liveTexiga näiteks, esmasel installil kaasa).

\TeX ile kuluks ära paketihaldur(nagu npm, mis tooks kaasa pakettide versioonihalduse ning sõltuvus hierarhia) ning viis integreerida tex failidega mingit programmeerimiskeelt("inline coding": javascript. ruby, kasvõi php). Viimast ideed kujutan hästi ette ka praegu rakendatavat - tex fail tuleb eelnevalt lihtsalt ühe korra veel läbi käia vastava keele parseri või interpretaatoriga.

Sai mõni lause rohkem kui paar.
