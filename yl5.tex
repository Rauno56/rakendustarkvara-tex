\newcommand{\tooLong}[2]{%
\StrLen{#1}[\theLength]%
\ifthenelse{ \theLength > #2 }{%
  \StrLeft{#1}{#2} \dots%
}{%
  #1%
}%
}

\newcommand{\weekDays}{
\foreach \d/\a in {esmaspäev/E, teisipäev/T, kolmapäev/K, neljapäev/N, reede/R, laupäev/L, pühapäev/P}{
  \noindent\d(\a)\\
}
}

\section{Ülesanne 5.}

\subsection{Parameetriga makro}

Eelmise ülesande lahendamiseks defineeritud taandamise makro:
\begin{displaymath}
  \cOut{90}{123} \quad \text{ja} \quad \cOutB{321}{123}
\end{displaymath}

\subsection{Hargnemisega makro}

Käsk, mis lühendab teksti maksimmalselt etteantud pikkuseks. Kui teksti lühendatakse märgitakse see ära kolme punktiga:

Parameetritega '123456' ja 5: \tooLong{123456}{5}

Parameetritega '1234' ja 5: \tooLong{1234}{5}

\subsection{Tsükliga makro}

Makro nädalapäevade välja trükkimiseks. Ülikasulik kui on vaja korduvalt igal pool nädalapäevi välja trükkida(produktiivsusgarantii):

\weekDays
